%% start of file `template.tex'.
%% Copyright 2006-2013 Xavier Danaux (xdanaux@gmail.com).
%
% This work may be distributed and/or modified under the
% conditions of the LaTeX Project Public License version 1.3c,
% available at http://www.latex-project.org/lppl/.


\documentclass[11pt,a4paper,sans]{moderncv}        % possible options include font size ('10pt', '11pt' and '12pt'), paper size ('a4paper', 'letterpaper', 'a5paper', 'legalpaper', 'executivepaper' and 'landscape') and font family ('sans' and 'roman')

% moderncv themes
\moderncvstyle{classic}                             % style options are 'casual' (default), 'classic', 'oldstyle' and 'banking'
\moderncvcolor{blue}                               % color options 'blue' (default), 'orange', 'green', 'red', 'purple', 'grey' and 'black'
%\renewcommand{\familydefault}{\sfdefault}         % to set the default font; use '\sfdefault' for the default sans serif font, '\rmdefault' for the default roman one, or any tex font name
%\nopagenumbers{}                                  % uncomment to suppress automatic page numbering for CVs longer than one page

% character encoding
\usepackage[utf8]{inputenc}                       % if you are not using xelatex ou lualatex, replace by the encoding you are using
%\usepackage{CJKutf8}                              % if you need to use CJK to typeset your resume in Chinese, Japanese or Korean

% adjust the page margins
\usepackage[scale=0.80]{geometry}
\setlength{\hintscolumnwidth}{3.13cm}                % if you want to change the width of the column with the dates
%\setlength{\makecvtitlenamewidth}{10cm}           % for the 'classic' style, if you want to force the width allocated to your name and avoid line breaks. be careful though, the length is normally calculated to avoid any overlap with your personal info; use this at your own typographical risks...

% personal data

\name{MohammadAmin}{Haghpanah}
% \title{Resumé title}                               % optional, remove / comment the line if not wanted
% \address{street and number}{postcode city}{country}% optional, remove / comment the line if not wanted; the "postcode city" and and "country" arguments can be omitted or provided empty
\phone[mobile]{+98~(937)~403~9221}                   % optional, remove / comment the line if not wanted
% \phone[fixed]{+2~(345)~678~901}                    % optional, remove / comment the line if not wanted
% \phone[fax]{+3~(456)~789~012}                      % optional, remove / comment the line if not wanted
\email{mdan.hagh@gmail.com}                               % optional, remove / comment the line if not wanted
% \homepage{github.com/AminHP}                         % optional, remove / comment the line if not wanted
\social[github][github.com/AminHP]{https://github.com/AminHP}
\extrainfo{Born on 5 May 1996}                 % optional, remove / comment the line if not wanted
\photo[80pt][0.4pt]{Picture}
% \photo[64pt][0.4pt]{picture}                       % optional, remove / comment the line if not wanted; '64pt' is the height the picture must be resized to, 0.4pt is the thickness of the frame around it (put it to 0pt for no frame) and 'picture' is the name of the picture file
\quote{Try to pay rather than getting paid}                                 % optional, remove / comment the line if not wanted

% to show numerical labels in the bibliography (default is to show no labels); only useful if you make citations in your resume
%\makeatletter
%\renewcommand*{\bibliographyitemlabel}{\@biblabel{\arabic{enumiv}}}
%\makeatother
%\renewcommand*{\bibliographyitemlabel}{[\arabic{enumiv}]}% CONSIDER REPLACING THE ABOVE BY THIS

% bibliography with mutiple entries
%\usepackage{multibib}
%\newcites{book,misc}{{Books},{Others}}
%----------------------------------------------------------------------------------
%            content
%----------------------------------------------------------------------------------
\begin{document}
%\begin{CJK*}{UTF8}{gbsn}                          % to typeset your resume in Chinese using CJK
%-----       resume       ---------------------------------------------------------
\makecvtitle

\section{Education}
\cventry{2019 -- Present}{M.Sc. in Artificial Intelligence}{University of Tehran}{Tehran, Iran}{}{}
\cventry{2014 -- 2019}{B.Sc. in Computer Software Engineering with a concentration on AI}{\newline Iran University of Science and Technology}{Tehran, Iran}{\newline GPA: 17.31/20 = 3.65/4}{}
\cventry{2010 -- 2014}{Diploma in Mathematics and Physics Discipline}{\newline Allameh Helli 3 High School}{Tehran, Iran}{\newline GPA: 19.77/20 = 4/4}{Affiliated with the National Organization for Development of Exceptional Talents}
\cventry{2007 -- 2010}{Middle School Degree}{Allameh Helli 2 Middle School}{Tehran, Iran}{}{Affiliated with the National Organization for Development of Exceptional Talents}

\section{Industrial Experience}
\cventry{2015 -- Present}{Owner}{\href{https://github.com/koala-team}{\color{blue}{Koala Team}}}{Tehran, Iran}{}{}

\cventry{2017 -- Present}{Co-founder}{\href{https://chillinwars.ir}{\color{blue}{ChillinWars}}}{Tehran, Iran}{}{}

\cventry{10/2017 -- 10/2018}{Senior Backend Developer (C\#)}{Green and Silver Leaves}{Tehran, Iran}{\newline I was working on the Mobayyen project. It's a big project and I've developed its web service from scratch. Also, I've improved most parts of the project's framework}{}

\cventry{07/2016 -- 09/2016}{Backend Developer (Python)}{\href{http://gandom.co}{\color{blue}{Gandom}}}{Tehran, Iran}{\newline I was working on the ChiChiKoo project (a service similar to foursquare) and I developed a RESTful API with Flask}{}

\cventry{08/2015 -- 09/2015}{Backend Developer (Python)}{Bajiru}{Tehran, Iran}{\newline Bajiru was a startup in the restaurant management area. I was working on a RESTful Flask API as a trainee}{}

\section{Teaching Experience}
\subsection{Teaching Assistant}
\cventry{Fall 2018}{Software Engineering (Dr. Mehrdad Ashtiani)}{\newline Iran University of Science and Technology}{Tehran, Iran}{}{}
\cventry{Spring 2018}{Computational Intelligence (Dr. Naser Mozayani)}{\newline Iran University of Science and Technology}{Tehran, Iran}{}{}
\cventry{Spring 2018}{Advanced Programming of C\# (Dr. Sauleh Etemadi)}{\newline Iran University of Science and Technology}{Tehran, Iran}{}{}
\cventry{Spring 2016}{Advanced Programming of C++ (Dr. Zeinab Movahedi)}{\newline Iran University of Science and Technology}{Tehran, Iran}{}{}


\section{Volunteer Experience}
\cventry{08/2019 -- Present}{Technical Manager and Game Developer of \href{https://2020.chillinwars.ir}{\color{blue}{ChillinWars 2020}}}{\newline Iran University of Science and Technology}{Tehran, Iran}{}{}

\cventry{08/2018 -- 03/2019}{Supervisor of \href{https://2019.chillinwars.ir}{\color{blue}{ChillinWars 2019}}}{\newline Iran University of Science and Technology}{Tehran, Iran}{}{}

\cventry{09/2017 -- 01/2018}{Chief and Technical Manager of \href{https://2017.chillinwars.ir}{\color{blue}{ChillinWars 2017}}}{\newline Iran University of Science and Technology}{Tehran, Iran}{}{}

\cventry{06/2016 -- 05/2017}{Member of the Scientific Association of the Computer Engineering Department}{Iran University of Science and Technology}{Tehran, Iran}{}{}

\cventry{10/2015 -- 01/2016}{Chief Manager and Organizer of a local ACM contest in the University}{\newline Iran University of Science and Technology}{Tehran, Iran}{}{}

\section{Skills}
\cvitem{Self Learning}{I think this is the most important skill of mine and I've learned my other skills by it.}
\cvitem{Programming}{
	\textit{Proficient at:} Python, C\#, C++, C, R, Erlang, Java, UML \newline
	\textit{Familiar with:} JavaScript, HTML, MATLAB, VHDL, Assembly, CSS, Pascal, PHP
}
\cvitem{Framework/Library}{Django REST, ASP.NET, Flask, Keras, Tensorflow, Keras-RL, OpenCV, PyGame, SDL, ggplot}
\cvitem{Tool}{Qt, PyQt, Visual Paradigm, Telegram Client, Telegram Bot, Android Studio, CUDA, ANTLR, Xilinx ISE}
\cvitem{Database}{SQL, MongoDB, Riak, Redis}
\cvitem{Project and Code Management Tool}{Git, TFS, Trello}
\cvitem{Other}{Linux, Docker, NGINX}


\section{Selected Projects}
\cvitem{2020 -- Present}{\textbf{drf-psq:} An extension for Django REST framework that gives support for having view-based \textit{permission\_classes}, \textit{serializer\_class}, and \textit{queryset}. \href{https://github.com/drf-psq/drf-psq}{\color{blue}{Github link}}.}

\cvitem{2019 -- Present}{\textbf{AnyTrading:} A collection of \href{https://gym.openai.com}{\color{blue}{OpenAI Gym}} environments for reinforcement learning-based trading algorithms with a great focus on simplicity, flexibility, and comprehensiveness. \href{https://github.com/AminHP/gym-anytrading}{\color{blue}{Github link}}.}

\cvitem{2018 -- Present}{\textbf{Musical Chord Detection:} An application that detects \href{https://en.wikipedia.org/wiki/Musical_note}{\color{blue}{musical notes}} in a musical signal (created by Piano, Guitar, etc). It's a very difficult problem and still isn't solved completely. Typically there exist 108 different notes and detecting a single note is kind of simple but the problem shows up when some notes are played simultaneously (chords). Imagine if someone plays 10 Piano notes with his 10 fingers, then there could be almost $100^{10}$ possible different chords. Solving this problem using basic ANN algorithms is not actually possible.}

\cvitem{2017 -- Present}{\textbf{Chillin:} A tool for creating game AI competitions. It consists of multiple components, including a \href{https://github.com/koala-team/Chillin-PyServer}{\color{blue}{server framework}} written in Python, four components written in \href{https://github.com/koala-team/Chillin-PyClient}{\color{blue}{Python}}, \href{https://github.com/koala-team/Chillin-CSharpClient}{\color{blue}{C\#}}, \href{https://github.com/koala-team/Chillin-CppClient}{\color{blue}{C++}}, and \href{https://github.com/koala-team/Chillin-JavaClient}{\color{blue}{Java}}. Also, Chillin came up with a \href{https://github.com/koala-team/Chillin-Monitor-2}{\color{blue}{3D monitor}} to spectate the games and watch what happens in the field. \newline \href{https://2017.chillinwars.ir}{\color{blue}{ChillinWars 2017}}, \href{https://2019.chillinwars.ir}{\color{blue}{ChillinWars 2019}}, and \href{https://2020.chillinwars.ir}{\color{blue}{ChillinWars 2020}} utilized this tool to create games for their competitions. Some examples can be found \href{https://github.com/Chillin-Examples}{\color{blue}{here}}.}

\cvitem{2017 -- Present}{\textbf{Koala Serializer:} A tool similar to \href{https://developers.google.com/protocol-buffers}{\color{blue}{Google Protobuf}} that enables the Chillin framework to be much more automated and simpler. \href{https://github.com/koala-team/serializer}{\color{blue}{Github link}}.}

\cvitem{2019}{\textbf{InstaRobot:} An \href{https://cafebazaar.ir/app/com.instagram.instarobot/?l=en}{\color{blue}{Android application}} that provides some tools for Instagram users.}

\cvitem{2018}{\textbf{Facial Expression Recognition (Bachelor Final Project):} A new solution for solving the famous Facial Expression Recognition problem using MLP and feature extraction. It detects 8 emotions (anger, contempt, disgust, fear, happiness, neutral, sadness, surprise) with an average accuracy of 97\% on the CK+ dataset.  \href{https://www.dropbox.com/s/i6pkzhfzp8jkpaz/FinalProjectArticle_FER.pdf?dl=0}{\color{blue}{Article link}}.}

\cvitem{Spring 2017}{\textbf{Inverted Pendulum:} A system that simulates the famous \href{https://en.wikipedia.org/wiki/Inverted_pendulum}{\color{blue}{Inverted Pendulum}} problem written in Python. Also, a fuzzy controller is implemented that tries to keep the pendulum inverted in the environment. It was a project for the Computational Intelligence course. \href{https://github.com/AminHP/fuzzy_inverted_pendulum}{\color{blue}{Github link}}.}

\cvitem{Fall 2016}{\textbf{Pourse:} A web service and application that provides some information about Stocks for Software Engineering course project. The project's backend was implemented using Erlang and Python programming languages and the Riak database.  \href{https://gitlab.com/Pourse/Server}{\color{blue}{Gitlab link}}.}

\cvitem{2015 -- 2017}{\textbf{IJust:} An open-source online ACM judge. \href{https://github.com/koala-team/ijust_server}{\color{blue}{Github link}}, \href{https://ijust.ir}{\color{blue}{ijust.ir}}.}

\cvitem{Spring 2015}{\textbf{OCR Site:} A simple website that provides single character OCR written in Python for Advanced Programming course project. The recognizer was an MLP and all parts of it were implemented from scratch (including ANN trainer).  \href{https://github.com/AminHP/OCR-Site}{\color{blue}{Github link}}.}

\cvitem{Fall 2014}{\textbf{PyTanks:} A multiplayer network game implemented by Python for the Basic Programming course project. Also, an AI (using Q-Learning) was implemented for tanks to help them run away from bombs. \href{https://github.com/mak-elmos/PyTanks}{\color{blue}{Github link}}.}

\cvitem{2011 -- 2013}{\textbf{Robot Path Planning:} RoboCup 3D Soccer Simulation is a seniors' tournament that is a part of robotics tournaments such as IranOpen. Its goal is to write a code that manages 11 \href{https://en.wikipedia.org/wiki/Nao_(robot)}{\color{blue}{NAO robots}} to play soccer in a simulated environment. My job in the team was creating and implementing a new path planning algorithm for these robots. Our TDP which was sent for World Championship 2013 competitions and qualified, can be found \href{https://www.dropbox.com/s/nkwcxfojufvafgt/3d_tdp_s4m.pdf?dl=0}{\color{blue}{here}}.  \href{https://gitlab.com/4/4d-simulation}{\color{blue}{Gitlab link}}.}

\cvitem{2011 -- 2012}{\textbf{Car Tracking:} A dynamic system that detects cars and their speeds by image processing and checks whether the cars move between highway lines. \href{https://github.com/AminHP/CarTracking}{\color{blue}{Github link}}.}

\cvitem{2011 -- 2012}{\textbf{Inverted Pendulum:} A system that simulates the famous \href{https://en.wikipedia.org/wiki/Inverted_pendulum}{\color{blue}{Inverted Pendulum}} problem written in C++. Also, an AI is implemented that uses Q-Learning to keep the pendulum inverted in the environment. \href{https://github.com/AminHP/InvertedPendulum}{\color{blue}{Github link}}.}

\cvitem{2010 -- 2011}{\textbf{Othello:} An object-oriented client/server platform that provides an interface for Othello AI programs are written in C++. Also, an AI was implemented using the Minimax algorithm.}

\pagebreak

\section{Github Contributions}
\cvitem{\href{https://github.com/rsquaredacademy/olsrr}{\color{blue}olsrr}}{ Added stepwise selection algorithms based on adjusted R-squared metric. \href{https://github.com/rsquaredacademy/olsrr/pull/163}{\color{blue}PR link.}}

\cvitem{\href{https://github.com/openai/gym}{\color{blue}OpenAI-Gym}}{ Added gym-anytrading. \href{https://github.com/openai/gym/commit/bce79003d09daae4905b5e092b58ba0c5dc9abc1}{\color{blue}Commit link.}}

\cvitem{\href{https://github.com/rochacbruno/flasgger}{\color{blue}flasgger}}{Added "importing other spec files" support. \href{https://github.com/flasgger/flasgger/commit/99bf27f16b48bdb9819a101d94ce2b9fa7da2bbf}{\color{blue}Commit link.}}

\cvitem{\href{https://github.com/cudamat/cudamat}{\color{blue}cudamat}}{Added correlate function. \href{https://github.com/cudamat/cudamat/commit/931ef9234e521a222206be2dcb889f8459f9c164}{\color{blue}Commit link.}}


\section{Selected Master Courses}
\subsection{University of Tehran, Iran}
\cventry{Winter 2019}{Reinforcement Learning}{}{\newline Instructor: \href{https://ece.ut.ac.ir/en/~mnili}{\color{blue}{Dr. Majid Nili Ahmadabadi}}}{\textit{Grade: 20/20}}{}

\cventry{Winter 2019}{Pattern Recognition}{}{\newline Instructor: \href{https://ece.ut.ac.ir/en/~araabi}{\color{blue}{Dr. Babak Nadjar Araabi}}}{\textit{Grade: A}}{}

\cventry{Winter 2019}{Data Analytics}{}{\newline Instructor: \href{https://ece.ut.ac.ir/en/~asadeghi}{\color{blue}{MohamadAmin Sadeghi}}}{\textit{Grade: $A^+$}}{}

\cventry{Fall 2020}{Statistical Inference}{}{\newline Instructor: \href{https://ece.ut.ac.ir/en/~bahrak}{\color{blue}{Behnam Bahrak}}}{\textit{Grade: 20/20}}{}


\section{Online Courses}
\subsection{Reinforcement Learning - University of Alberta}
\cventry{04/23/2020}{1 - Fundamentals of Reinforcement Learning}{}{\href{https://coursera.org/verify/Q92PSPE2HU94}{\textcolor{blue}{Certificate}}}{\textit{Score: 100/100}}{}
\cventry{04/24/2020}{2 - Sample-based Learning Methods}{}{\href{https://coursera.org/verify/EQR9E5MZZ3EH}{\textcolor{blue}{Certificate}}}{\textit{Score: 100/100}}{}
\cventry{04/25/2020}{3 - Prediction and Control with Function Approximation}{}{\newline \href{https://coursera.org/verify/5R4YZVMUSM63}{\textcolor{blue}{Certificate}}}{\textit{Score: 100/100}}{}
\cventry{04/26/2020}{4 - A Complete Reinforcement Learning System (Capstone)}{}{\newline \href{https://coursera.org/verify/99DFMU587VNV}{\textcolor{blue}{Certificate}}}{\textit{Score: 100/100}}{}

\subsection{Deep Learning - deaplearning.ai}
\cventry{05/08/2020}{1 - Neural Networks and Deep Learning}{}{\href{https://coursera.org/verify/8TVRPP3AQ8WA}{\textcolor{blue}{Certificate}}}{\textit{Score: 100/100}}{}
\cventry{05/09/2020}{2 - Improving Deep Neural Networks}{}{\href{https://coursera.org/verify/62PCW58LV6TV}{\textcolor{blue}{Certificate}}}{\textit{Score: 100/100}}{}
\cventry{06/03/2020}{3 - Structuring Machine Learning Projects}{}{\href{https://coursera.org/verify/CG6GYMBS2CXS}{\textcolor{blue}{Certificate}}}{\textit{Score: 100/100}}{}

\section{Awards \& Honors}
\subsection{Iran University of Science and Technology}
\cventry{2014 -- 2019}{Software Engineering field}{}{\newline Ranked Second Place for my total average among all entrants of the year 2014}{}{}
\cventry{2014 -- 2019}{Artificial Intelligence field}{}{\newline Ranked Second Place for my total average among all entrants of the year 2014}{}{}
\cventry{2016 -- 2017}{Software Engineering field}{}{\newline Ranked Third Place among top students of the year}{}{}
\cventry{May 2017}{Harekat Ceremony}{}{\newline Ranked First Place in a competition among all scientific associations of the university}{}{}
\cventry{2014 -- 2015}{Software Engineering field}{}{\newline Ranked Second Place among top students of the year}{}{}

\pagebreak

\subsection{Allameh Helli 3 High School}
\cventry{Apr 2013}{RoboCup IranOpen 2013}{\newline 3D Soccer Simulation League}{Certificate of Participation}{}{}
\cventry{Mar 2013}{RoboCup World Championship 2013}{\newline 3D Soccer Simulation League}{Qualified}{}{}
\cventry{Feb 2013}{Farzanegan RoboCup 2013}{\newline 3D Soccer Simulation League}{Ranked First Place}{}{}
\cventry{Apr 2012}{RoboCup DutchOpen 2012}{\newline 3D Soccer Simulation League}{Certificate of Participation}{}{}
\cventry{Apr 2012}{RoboCup IranOpen 2012}{\newline 3D Soccer Simulation League}{Certificate of Participation}{}{}
\cventry{2012}{Seminar on Science and Technology}{}{Ranked First Place}{}{}

\subsection{Allameh Helli 2 Middle School}
\cventry{2010}{Seminar on Science and Technology}{}{Ranked First Place}{}{}

\section{Interests}
\cvlistdoubleitem{Machine Learning}{Machine Vision}
\cvlistdoubleitem{Reinforcement Learning}{Signal Processing}
\cvlistdoubleitem{Bioinformatics}{Game Development}
\cvlistdoubleitem{Artificial Neural Networks}{Robotics}
\cvlistdoubleitem{Project and Team Management}{Software Engineering}
\cvlistdoubleitem{Object Oriented Programming}{Functional Programming}
\cvlistdoubleitem{Dota2}{Foosball}

\section{Personality}
\cvlistitem{\href{https://mycreativetype.com/type/maker}{\color{blue}{https://mycreativetype.com/type/maker}}}
\cvlistitem{\href{https://www.16personalities.com/istj-personality}{\color{blue}{https://www.16personalities.com/istj-personality}}}

%\clearpage\end{CJK*}                              % if you are typesetting your resume in Chinese using CJK; the \clearpage is required for fancyhdr to work correctly with CJK, though it kills the page numbering by making \lastpage undefined
\end{document}


%% end of file `template.tex'.
